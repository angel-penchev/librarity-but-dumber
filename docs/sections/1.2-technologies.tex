\section{Технологии и развойни среди използвани за реализация на проекта}
\subsection{C++}
С++ е език за програмиране от високо ниво, който има възможности за програмиране на големи програмни системи, с високо ниво на функционалност, представяне и ефикасност. Той е обектно ориентиран език със статични типове. Езикът се използва за изработката на проекта.

\subsection{CMake}
CMake е безплатен софтуер с отворен код за автоматизация на компилиране, тестване, пакетиране и инсталиране на софтуер чрез използване на независим от компилатора метод. CMake не е система за компилиране, а такава, която генерира компилационни файлове на други системи, като Make, Qt Creator, Ninja, Microsoft Visual Studio и т.н. Инструментът се използва за пакетиране на програмата разработена в проекта.

\subsection{gTest}
Google Test е библиотека с отворен код за тестване на модули за езика C++. В проекта се използва за създаването и изпълняването на модулни тестове, интеграционни тестове и всеобхващащи (на англ. "end-to-end") тестове.

\subsection{Doxygen}
Doxygen е система за анализ и генериране на документация от коментари в изходния програмен код на даден проект. използвана е за генерацията на част от документацията, приложена към проекта.

\subsection{Git}
Git е децентрализирана система за контрол на версиите на файлове. Използва се за управление на версиите на проекта.

\subsection{GitHub}
GitHub e уеб базирана услуга разпространяване на софтуерни проекти, съвместни разработкa върху тях и т.н. Проектите се съхраняват в т.нар. хранилища. Платформата се базира на Git системата за контрол и управление на версиите. Проектът се съхранява в GitHub кодово хранилище на адрес: \url{https://github.com/angel-penchev/librarity-but-dumber}.

\subsection{GitHub Actions}
GitHub Actions е платформа за непрекъсната интеграция и непрекъсната доставка (CI/CD), която позволява автоматизиране на пакетирането и тестването на проекти в платформата GitHub. В проекта се използва за изпълняването на тестове при качване в хранилището, както и за пакетирането на проекта при създване на нова версия.

\subsection{CLion}
CLion е кросплатформенан развойна среда за разработка на C/C++ проекти. CLion включва такива функции като интелигентен редактор, генериране на код, осигуряване на качеството на кода, автоматизирано рефакториране, анализ на код, мениджмънт на проекти, интегрирани системи за контрол на версиите и дебъгер. CLion беше използвана за генерирането на структурата на проекта и разработката му.