\section{Настройка на среда за разработка, компилиране и изпълнение}
За да се настрои среда за разработка на проекта, трябва да се изпълнят следните стъпки:
\begin{enumerate}
    \item Да се изтегли кодовото хранилище на проекта от платформата GitHub.
    \lstinputlisting[language=bash]{code/clone-repo.sh}

    \item Да се компилира проекта на проекта. Това може да бъде направено като се изпълни следната командата:
    \lstinputlisting[language=bash]{code/build-make.sh}
    или ако инструмента Make е недостъпен:
    \lstinputlisting[language=bash]{code/build.sh}

    \item Да се изпълни проекта. Това може да бъде направено като се изпълни следната командата:
    \lstinputlisting[language=bash]{code/run-make.sh}
    или:
    \lstinputlisting[language=bash]{code/run.sh}

    За да се изпълнят тестовете на проекта, трябва да се изпълни следната командата:
    \lstinputlisting[language=bash]{code/test-make.sh}
    или:
    \lstinputlisting[language=bash]{code/test.sh}
\end{enumerate}