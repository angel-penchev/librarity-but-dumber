\section{Изисквания към реализацията на проекта}\
Да се напише програма, реализираща информационна система, която поддържа библиотека от електронни книги. Програмата съхранява и обработва данни за наличните в момента книги в текстов или двоичен файл във формат по ваш избор. Всяка книга се характеризира със следните данни:
\begin{itemize}
    \item автор
    \item заглавие
    \item име на текстов файл, където е записан текста на книгите
    \item кратко описание
    \item рейтинг
    \item международен стандартен номер на книга (ISBN)
\end{itemize}
Системата поддържа две нива на достъп: неоторизиран (за повечето операции) и оторизиран (за операции, които трябва да се изпълнят с администраторски права). Достъпът до оторизираните функции става при въвеждане на парола за администратор.

Информацията за книгите в библиотеката се пази в текстов или двоичен файл във формат по ваш избор. Работата с програмата се осъществява в диалогов режим, като се поддържат следните операции:

\begin{table}[!ht]
    \begin{tabular}{lcccccc}
        \toprule
        \textbf{Операция}             & \textbf{Описание}                                                                                                                                                                                                                                                                                                                                    & \textbf{Изисква оторизиран достъп?} \\ \midrule
        Добавяне на книга             & Добавя в библиотеката нова книга, като въвежда пълна информация за нея.                                                                                                                                                                                                                                                                              & Да                                  \\
        Премахване на книга           & Премахване на книга от библиотеката, като в този случай се дава възможност за се изтрие и файла, свързан с книгата.                                                                                                                                                                                                                                  & Да                                  \\
        Сортиране на списъка с книги  & Извежда последователно за всяка книга следната информация: \textless{}заглавие\textgreater{}, \textless{}автор\textgreater{}, \textless{}ISBN\textgreater{}. Изведеният списък да е сортиран възходящо или низходящо (по желание на потребителя) по едно от следните полета: заглавие, автор, рейтинг, избрани от потребителя                        & Не                                  \\
        Намиране на книга по критерий & Извежда на екрана подробна информация за книга с въведени от потребителя заглавие, автор, ISBN или част от описание. При първите три критерия се изисква точно съвпадение, а при описанието може въведеният от потребителя низ да се съдържа в описанието. Търсене на книга по зададен критерий да игнорира регистъра на буквите (малки или големи). & Не                                  \\
        Извеждане на книга            & Извежда на екрана съдържанието на текстовия файл, в който е записана книгата. Поддържа се режим на извеждане по страници (извеждане на зададен от потребителя брой редове и пауза след това) и режим на извеждане по изречения (извеждане на изречение до препинателен знак и пауза след това).                                                      & Не                                  \\ \bottomrule
    \end{tabular}
\end{table}